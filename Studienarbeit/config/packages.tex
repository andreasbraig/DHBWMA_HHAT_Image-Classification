% Pakete laden
\usepackage[utf8]{inputenc}
\usepackage[T1]{fontenc}
\usepackage[ngerman]{babel}
\usepackage{helvet} % Arial Schriftart
\renewcommand{\familydefault}{\sfdefault} % Setzt die Standardschriftart auf sans-serif (Arial)
\usepackage[printonlyused]{acronym}
\usepackage[onehalfspacing]{setspace}
\usepackage[left=2.5cm,right=2.5cm,bottom=4cm,top=2.5cm]{geometry}
\usepackage{ragged2e}
\usepackage{listings}
\usepackage{graphicx}
\usepackage{subcaption}
\usepackage{tabularx}
\usepackage{tikz}
\usepackage[style=ieee]{biblatex}
\usepackage{lipsum} % Dummy Text
\usepackage{tabularray}
\usepackage[x11names]{xcolor} % Für erweiterte Farboptionen
\usepackage{pdfpages}
\usepackage{float}
\usepackage{caption}
\usepackage{csquotes}
\usepackage{enumitem}
\usepackage{amssymb}
\usepackage[toc,page]{appendix}
\usepackage{pifont}
\usepackage{colortbl}
\usepackage{booktabs}
\usepackage{courier}
\usepackage[gen]{eurosym}
\usepackage{xcolor} % Falls noch nicht enthalten


\addbibresource{resources/directories/bibliography.bib}

% Automatisch eine Leerzeile nach einem Absatz einfügen
\setlength{\parskip}{\baselineskip}


% % Kopf- und Fußzeile
% \usepackage[headsepline=0.1pt, footsepline=0.1pt, automark]{scrlayer-scrpage}
% \pagestyle{scrheadings}
% \clearpairofpagestyles % Löscht voreingestellte Kopf- und Fußzeilen
% \ofoot[\pagemark]{\pagemark} % Seitenzahl mittig unten
% \ihead{\upshape\Vorname{} \Nachname{}} % Kopfzeile ohne Kursivschrift
% \ohead{\upshape\Art{}} % Kopfzeile ohne Kursivschrift

\usepackage[headsepline=0.1pt, footsepline=0.1pt, automark]{scrlayer-scrpage}
% Falls noch nicht enthalten, für Farben notwendig

\setkomafont{pageheadfoot}{\color{lightgray}} % Kopf- und Fußzeile grau einfärben

\pagestyle{scrheadings}
\clearpairofpagestyles % Löscht voreingestellte Kopf- und Fußzeilen

% Kopf- und Fußzeile grau einfärben
\ihead{\textcolor{lightgray}{\upshape\Vorname{} \Nachname{}}} % Kopfzeile ohne Kursivschrift
\ohead{\textcolor{lightgray}{\upshape\Art{}}} % Kopfzeile ohne Kursivschrift
\ofoot[\textcolor{lightgray}{\pagemark}]{\textcolor{lightgray}{\pagemark}} % Seitenzahl mittig unten in Grau

% Charter Schriftart für Kapitelüberschriften
%\setkomafont{disposition}{\normalfont\bfseries}

% Formatierung der Chapter-Überschriften
\makeatletter
\renewcommand\@makechapterhead[1]{%
	\vspace*{0\p@}%
	{\parindent \z@ \raggedright
		\normalfont
		\interlinepenalty\@M
		\fontsize{16pt}{16pt}\selectfont \bfseries \thechapter.\  #1\par\nobreak
		\vskip 10\p@
}}
\renewcommand\section{\@startsection {section}{1}{\z@}%
	{-2.5ex \@plus -1ex \@minus -.2ex}% Abstand vor der Überschrift
	{2.3ex \@plus.2ex}% Abstand nach der Überschrift
	{\normalfont\fontsize{14pt}{14pt}\selectfont\bfseries}}
\renewcommand\subsection{\@startsection {subsection}{2}{\z@}%
	{-1.25ex\@plus -1ex \@minus -.2ex}% Abstand vor der Überschrift
	{1.5ex \@plus .2ex}% Abstand nach der Überschrift
	{\normalfont\fontsize{12pt}{12pt}\selectfont\bfseries}}
\makeatother





% Abstand vor und nach Kapitelüberschriften
\renewcommand*{\chapterheadstartvskip}{\vspace*{1cm}}
\renewcommand*{\chapterheadendvskip}{\vspace{2\baselineskip}}

% Grafikpfad definieren
\graphicspath{{resources/images/}}

\setlength{\parindent}{0pt}

% Für Zitierung von URLs
\newcommand*{\quelle}{% 
	\footnotesize Quelle: 
}
\urlstyle{same}

\newcommand{\code}[1]{\texttt{#1}}


\definecolor{XMLgreen}{HTML}{697821}

\lstdefinelanguage{Kotlin}{
	morekeywords={class, fun, val, var, return, if, else, for, while, in, throw, try, catch},
	sensitive=true,
	comment=[l][\%],
	morecomment=[s][/*][*/],
	morestring=[b]"
}

% Listings-Style für Kotlin anpassen
\lstdefinestyle{kotlin}{
	language=Kotlin,
	basicstyle=\ttfamily\scriptsize,
	numbers=left,
	numberstyle=\tiny\color{gray},
	breaklines=true,
	captionpos=b,
	frame=single,
	keywordstyle=\color{blue}\bfseries,
	commentstyle=\color{green},
	stringstyle=\color{red},
	showstringspaces=false,
	tabsize=4,
	morekeywords={@Composable, fun, val, var, Box, Modifier, Alignment, Text} % Weitere Schlüsselwörter hinzufügen
}

\lstdefinelanguage{json}{
	morekeywords={true, false, null},
	sensitive=false,
	comment=[l][//],
	morestring=[b]",
}

% Listings-Style für JSON anpassen
\lstdefinestyle{json}{
	language=json,
	basicstyle=\ttfamily\scriptsize,
	numbers=left,
	numberstyle=\tiny\color{gray},
	breaklines=true,
	captionpos=b,
	frame=single,
	stringstyle=\color{blue},
	keywordstyle=\color{red}\bfseries, % Schlüsselwörter in Rot und fett
	commentstyle=\color{gray}, % Kommentare in Grau
	showstringspaces=false,
	tabsize=4
}

% Definiere die benutzerdefinierte Umgebung für Code
\newenvironment{mycode}
{\captionsetup{type=listing}} % Setzt den Typ der Beschriftung auf "listing"
{} % Ende der Umgebung (kann leer bleiben)


%Für Links und Hyperlinks
\usepackage[colorlinks = true,
linkcolor = black,
urlcolor  = black,
citecolor = black]{hyperref}

%Eigenschaften des PDF Dokuments anpassen (Titel, Art der Arbeit, Autor)
\hypersetup{pdftitle={\Titel},
	pdfsubject={\Art},
	pdfauthor={{\Vorname} {\Nachname}}
}


