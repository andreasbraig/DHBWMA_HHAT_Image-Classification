\chapter{Problemstellung und Ziel dieser Arbeit} \label{chap:problemstellung_und_ziel_dieser_arbeit}

Die zunehmende Automatisierung industrieller Prozesse erfordert zuverlässige Qualitätskontrollsysteme, insbesondere in der Fertigung von elektronischen Baugruppen wie Printed Circuit Boards \acp{PCB}. Im letzten Semester wurde ein KI-basiertes System entwickelt, das mittels TensorFlow Convolutional Neural Networks \acp{CNN} 
Defekte auf \acp{PCB} erkennt. Dieses System basiert auf einer Online-Implementierung ohne lokale Anpassungsmöglichkeiten und Benutzeroberfläche. 

Aktuell bestehen drei zentrale Herausforderungen: Erstens bietet die Online- Implementierung keine lokale Kontrolle über Parameter oder Daten, was die Flexibilität in industriellen Umgebungen limitiert. Zweitens fehlt eine intuitive Schnittstelle zur Visualisierung von 
Klassifizierungsergebnissen oder Anpassung von Einstellungen, was die Benutzerinteraktion erschwert. Drittens soll die Leistung der bisher verwendeten \ac{CNN}-Architektur evaluiert werden und weitere Architekturen oder Optimierungstechniken sollen getestet werden, um die
Erkennungsgenauigkeit zu verbessern.  

Ziel dieser Arbeit ist es, die bestehende Lösung in eine lokale Anwendung zu überführen, die folgende Kernkomponenten integriert: Eine zentrale Parametrisierung via JSON ermöglicht die flexible Steuerung aller Modell- und Systemparameter, während eine 
modularisierte Codebasis die Wartbarkeit und Wiederverwendbarkeit des Python-Codes verbessert. Zusätzlich soll eine Webanwendung mit Python-application Programming Interface \ac{API} entwickelt werden, die eine Echtzeit-Darstellung von \acp{PCB}-Bildern, Klassifizierungsergebnissen und Systemstatus bietet. 
Parallel erfolgt eine systematische Modellre-Evaluation, bei der das aktuelle \ac{CNN} mit alternativen Architekturen oder Optimierungstechniken verglichen wird.

Durch diese Maßnahmen soll die Darstellung der industriellen Anwendbarkeit im FESTO CP Lab gestärkt werden. Die neue Webvisualisierung soll greifbar machen, was bildverarbeitende Qualitätskontrolle bedeutet, indem sie die Echtzeit-Analyse und -Ergebnisse der \acp{PCB}-Bilder anschaulich darstellt. Benutzer können direkt sehen, dass Defekte erkannt und klassifiziert werden, was die Transparenz und Nachvollziehbarkeit des gesamten Prozesses erhöht.