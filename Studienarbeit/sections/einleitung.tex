\chapter{Problemstellung und Ziel dieser Arbeit} \label{chap:problemstellung_und_ziel_dieser_arbeit}

Die zunehmende Automatisierung industrieller Prozesse erfordert zuverlässige Qualitätskontrollsysteme, insbesondere in der Fertigung von elektronischen Baugruppen wie \acp{PCB}. Im letzten Semester wurde in einer Machbarkeitsstudie untersucht ob ein KI-basiertes System umsetzbar ist, welches mittels TensorFlow \acp{CNN} 
Defekte auf \acp{PCB} erkennt. Dieses System basiert auf einer Online-Implementierung Benutzeroberfläche. 

Aktuell bestehen drei zentrale Herausforderungen: Erstens bietet die Online- Implementierung keine lokale Kontrolle über Parameter oder Daten, was die Flexibilität limitiert. Zweitens fehlt eine intuitive Schnittstelle zur Visualisierung von 
Klassifizierungsergebnissen, was die Benutzerinteraktion erschwert. Drittens soll die Leistung der bisher verwendeten \ac{CNN}-Architektur evaluiert werden und weitere Architekturen oder Optimierungstechniken sollen getestet werden, um die
Erkennungsgenauigkeit zu verbessern. Hierfür stehen eine Reihe neuer \ac{PCB}s zur Verfügung, die in einer zu dieser Arbeit parallelen Studienarbeit entwickelt wurden.

Ziel dieser Arbeit ist es, die bestehende Lösung in eine lokale Anwendung zu überführen, die folgende Kernkomponenten integriert: Eine zentrale Parametrisierung. Diese ermöglicht die flexible Steuerung aller Modell- und Systemparameter, während eine 
modularisierte Codebasis die Wartbarkeit und Wiederverwendbarkeit des Python-Codes verbessert. Zusätzlich soll eine Webanwendung mit einer Python \ac{API} entwickelt werden, die eine Darstellung in echtzeit von \acp{PCB} Klassifizierungsergebnissen bietet. 
Parallel erfolgt eine systematische Modellre-Evaluation, bei der das aktuelle \ac{CNN} mit alternativen Architekturen oder Optimierungstechniken verglichen wird.

Durch diese Maßnahmen soll die Darstellung der industriellen Anwendbarkeit im FESTO CP Lab gestärkt werden. 
Die neue Webvisualisierung soll greifbar machen, was bildverarbeitende Qualitätskontrolle bedeutet, indem sie in echtzeit Analyse und Ergebnisse der \acp{PCB} anschaulich darstellt. Benutzer können direkt sehen, dass Defekte erkannt und klassifiziert werden, was die Transparenz und Nachvollziehbarkeit des gesamten Prozesses erhöht.