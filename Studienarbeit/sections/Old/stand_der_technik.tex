\chapter{Analyse bestehender Bewässerungslösungen}

Wie bereits in der Einleitung hingeführt, konnte der Bereich Smart Home in den letzten Jahren schnell sowohl aufgrund der hohen Anfrage aber auch aufgrund des schnellen Wandels der Technologie expandieren. Vorteile wie Datenüberwachung und der Automatisierung von Prozessen, durch die häufig eine Effizienzsteigerung resultiert, wurde das Prinzip von Smart Home auch in den Industriellen Sektoren angepasst. „Smart“ steht somit stellvertretend für die Eigenschaften der neuen Möglichkeiten. Zusätzlich wird dem Begriff jeweils ein Sektor zugewiesen. So werden beispielsweise Technologien in der Industrie unter dem Begriff „Smart Factrory“ oder in der Landwirtschaft unter dem Begriff „Smart Farming“ zugeordnet. In diesem Absatz werden neue Innovationen des „Smart Farmings“ sowie des „Smart Homes“ unter Betrachtung des Aspekts der Pflanzenbewässerung vorgestellt. Abschließend wird eine Abgrenzung der Lösung der Studienarbeit zu bereits bestehenden Projekten hergestellt. 

\section{Aktuelle Technologien im Bereich Smart Farming}
In der Agrar-Industrie stellt sich „Smart Farming“ oder auch „Precision Farming“ als das Werkzeug zur örtlichen und zeitlichen Optimierung heraus. Mit der Digitalisierung der Landwirtschaft mittels Bewässerungssystemen oder innovativen Erntemaschinen rückt die Überwachung der Pflanzen immer mehr in den Fokus \cite{fingerPrecisionFarmingNexus2019}. Mittels Sensoren wie beispielsweise zur Erfassung der Bodenfeuchte oder Beschaffenheit kann die gesamte Ernte dauerhaft überwacht und dadurch die optimalen Bedingungen für unterschiedliche Pflanzen sichergestellt werden. Der Einsatz von neue Technologien ermöglicht eine Erstellung von 3D-Bodenkarten zur Ermittlung der Bearbeitungstiefe. Zusätzlich können die Daten in Echtzeit über Cloud-Dienste für die Landwirte jederzeit einsehbar sein. Gerade der Aspekt der Datenerfassung über die Cloud löst Probleme wie Harmonisierungsprobleme der Datenformate und ermöglicht ein Outsourcen der Datenverarbeitung über Dienstleister \cite[S. 33 ff.]{windischAgrarwissenschaftlichesSymposiumHans2016}. Auf weitere Aspekte wie \ac{ki}-Unterstützung und Datensicherheit wird Verlauf im Bericht nicht tiefer eingegangen. 

\section{Smart Home Systeme zur Bewässerung von Pflanzen}
Auch im Bereich Smart Home gibt es bereits Lösungen, um die eigenen Zimmerpflanzen ohne großen Aufwand am Leben zu erhalten. Durch Automatisierung entfällt beispielsweise das Pflegen von Gießplänen sowie die direkte Bewässerung. Durch eine Gegenüberstellung vorhandener Produkte in diesem Gebiet sollen die Unterschiede der Systeme herausgearbeitet werden.

In \ref{tab:overview_existing_solutions} ist tabellarisch eine Übersicht der verschiedenen Bewässerungssysteme dargestellt. Hierfür wird von jeder Produktkategorie ein repräsentatives Produkt auf dem Markt ausgewählt und entsprechend der vom Hersteller angegebenen Daten ausgearbeitet. 
Bei dem System von Bluma handelt es sich um einen Tonkegel. Dieser gibt aufgrund der Materialbeschaffenheit nur Wasser an die Erde ab, wenn diese zu trocken wird und ist somit eine natürliche und energiefreie Lösung für eine automatische Bewässerung. Durch die Variante mit dem integrierten Schlauch befeuchtet sich das Material selbst über den angrenzenden Behälter. Es ist jedoch zu beachten, dass der Füllstand des Behälters von dem Anwender regelmäßig selbst zu überprüfen ist, um einen reibungslosen Gießzyklus zu gewährleisten. 

Produkte wie der selbstgießende Blumentopf von Lazy Leaf fungieren als selbstgesteuertes direktes Bewässerungssystem. Dabei verspricht der Hersteller eine automatisierte, im Blumentopf integrierte Pumpenlösung inklusive Wasserbehälter, bei dem die Pflanze entsprechend der eingestellten Timerfunktion und Gießmenge bewässert werden kann. Dieses Produkt besitzt 10 unterschiedliche Modi, mit denen alle gängigen Pflanzenbedürfnisse abgedeckt werden sollen. Dazu muss der Anwender die Grundbedürfnisse der zu bewässernden Pflanze einschätzen können, um eine richtige Auswahl des Modus zu treffen. Zudem ist auch hier eine automatische Meldung für das Nachfüllen des Wasserbehälters nicht gegeben, wodurch auch hier eine manuelle Überprüfung dessen notwendig ist.

Hinsichtlich der Sensortechnik gibt es in Bezug auf die Pflanzenbewässerungssysteme viele neue Technologien. Bei dem \ac{ki}-angebundenen, smarten Pflanzensensor von FYTA können die Daten wie beispielsweise Bodenfeuchte der zu überwachenden Pflanze mittels Bluetooth oder auch WLAN an eine App übermittelt werden um jederzeit Echtzeitdaten oder auch Langzeitdaten abrufen zu können. Mittels einer am Sensor angebrachten Kamera und einer \ac{ki}-auswertbaren Datenbank, können so Nährstoffmängel der Pflanze diagnostiziert werden. Der Gießvorgang ist bei diesem System nicht vorgesehen und ist entweder manuell oder durch eine andere Produktlösung durchzuführen.

Ein weiterer Trend der Pflanzenbewässerungssysteme sind die sogenannten \ac{diy}-Kits. Dabei handelt es sich um eine vorgefertigte Auswahl an Hardware Komponenten für den Anschluss und die Steuerung über einen Mikrocontroller. Diese Art der Lösung entspricht zum größten Teil der im Bericht angewendeten Lösung. Dabei ist anzumerken, dass für die Installation ein technisches Verständnis für die Verdrahtung der unterschiedlichen Komponenten, für die Programmierung des Mikrocontrollers sowie für die Kalibrierung der Sensoren und Aktoren von Nöten ist.

Bei der Untersuchung der unterschiedlichen Produkte ergibt sich die Fragestellung, warum keine bestehende Lösung zur Steuerung über den Mikrocontroller gibt. Dies lässt sich mittels einer Betrachtung des Objekts Zimmerpflanze feststellen. Die Individualität jeder einzelnen Pflanze und ihrer Umgebung kann nur schwer über ein bestehendes System ausgewertet werden. Dazu zählen zum einen Grundbedürfnisse der Pflanze, wie beispielsweise der Wasser- oder Nährstoffbedarf, die je nach Art und Größe der Pflanze andere Werte erwarten. Andere Faktoren sind die Größe des Blumentopfs oder die Dichte und Art der verwendeten Bodenerde. Aufgrund der hohen Abweichungen der einzelnen Faktoren kann ohne \ac{ki}-Unterstützung somit nur schwer eine allgemeine Lösung für alle Pflanzenarten geschaffen werden.


\section{Apps zur Pflanzenpflege und -bewässerung}

Im Bereich der Pflanzenpflege gibt es mittlerweile eine Vielzahl von Apps, die unterschiedliche Funktionen bieten, um Nutzern bei der Überwachung und Pflege ihrer Pflanzen zu helfen. Diese Apps lassen sich in verschiedene Kategorien einteilen, die sich nach ihren Kernfunktionen richten, wie z.B. Gieß-Erinnerungen, Pflanzendatenverwaltung, Pflanzenidentifikation und -diagnose sowie die Bereitstellung individueller Pflegeempfehlungen. \autoref{tab:app_compare} gibt einen Überblick über gängige Apps und ihre Hauptfunktionen.

\begin{table}[H]
	\centering
	\caption{Vergleich von Pflanzenpflege-Apps basierend auf ihren Hauptfunktionen \cite{arslan5BestenHandyApps}}
	\label{tab:app_compare}
	\begin{tabularx}{\textwidth}{|>{\raggedright\arraybackslash}p{0.15\textwidth}|X|X|}
		\hline
		\textbf{App}           & \textbf{Hauptfunktionen}                                                                & \textbf{Besondere Merkmale}                                           \\ \hline
		\textbf{Flower Care}    & Gieß- und Düngungspläne, Pflegeerinnerungen, Pflanzendatenverwaltung                    & Anpassung an spezifische Pflanzenbedürfnisse, Standortverwaltung     \\ \hline
		\textbf{HelloPlant}     & Pflanzenscan, Pflegeempfehlungen, Bildanalyse für Gesundheitsdiagnose                   & Erkennung von Schädlingen/Krankheiten, Fotoanalyse                   \\ \hline
		\textbf{Planta}         & Pflegeplanung, Gieß- und Düngungserinnerungen, Umtopfen-Erinnerungen                     & Detaillierte Pflegepläne für spezifische Pflanzenarten               \\ \hline
		\textbf{PlantSnap}      & Pflanzenarten-Identifikation per Foto, große Datenbank für Pflanzenarten                & Sofortige Rückmeldung zur Pflanzenidentifikation                      \\ \hline
		\textbf{Gardenize}      & Gartenverwaltung, visuelles Tagebuch, langfristige Pflegeüberwachung                     & Langfristige Dokumentation des Pflanzenwachstums                      \\ \hline
	\end{tabularx}
\end{table}



\section{Vergleich bestehender Lösungen mit dem eigenen System zur Pflanzenbewässerung über Arduino und App}

Die Analyse bestehender Lösungen im Bereich der Pflanzenbewässerung hat gezeigt, dass viele der gängigen Systeme entweder auf einfachen mechanischen Prinzipien basieren oder eine eingeschränkte Anpassungsfähigkeit aufweisen. Die meisten bestehenden Produkte, wie der selbstgießende Blumentopf von Lazy Leaf oder die Tonkegel-Systeme von Bluma, bieten grundlegende automatische Bewässerungsfunktionen, die jedoch häufig auf festen Programmen oder manuellen Eingriffen, wie das händische Gießen, basieren. Diese Systeme bieten keine Integration mit modernen Technologien oder eine umfassende Anpassung an die individuellen Bedürfnisse der Pflanzen.

Im Gegensatz zu diesen herkömmlichen Systemen bietet das hier entwickelte System eine umfassende Lösung, die durch die Kombination von Arduino-basierter Automatisierung und einer mobilen App eine, bis auf das Auffüllen des Tanks, komplett automatisierte Pflanzenpflege ermöglicht. Das System nutzt Sensoren, um Daten zu erfassen, wie z.B. die Bodenfeuchtigkeit, und kann auf Basis dieser Daten Entscheidungen treffen, um die Bewässerung automatisch anzupassen. Dies wird durch eine App unterstützt, die dem Benutzer die Möglichkeit gibt, die Bewässerung zu überwachen und bei Bedarf anzupassen.

