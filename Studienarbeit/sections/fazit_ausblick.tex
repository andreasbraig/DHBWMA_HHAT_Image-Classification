
\chapter{Fazit und Ausblick} \label{chap:fazit_und_ausblick}

In dieser Arbeit wurde ein Ansatz zur Bildklassifikation mittels Convolutional Neural Networks (CNNs) untersucht und implementiert. Die Ergebnisse zeigen, dass CNNs eine hohe Genauigkeit bei der Klassifikation von Bildern erreichen können, insbesondere wenn sie auf spezifische Datensätze und Aufgaben zugeschnitten sind. Die durchgeführten Experimente haben die Bedeutung der Datenvorverarbeitung und der Hyperparameter-Optimierung hervorgehoben, um die Leistung des Modells zu maximieren.

Ein wesentlicher Bestandteil dieser Arbeit war die Entwicklung und Implementierung eines Custom CNN, das speziell auf die Anforderungen des gegebenen Datensatzes abgestimmt wurde. Die erzielten Ergebnisse bestätigen, dass maßgeschneiderte Architekturen, die auf die Besonderheiten der Daten eingehen, zu einer verbesserten Klassifikationsgenauigkeit führen können.

Trotz der erzielten Fortschritte gibt es mehrere Bereiche, die für zukünftige Arbeiten von Interesse sein könnten. Eine mögliche Erweiterung besteht darin, die Integration des Custom CNN in eine umfassendere Programmpipeline zu ermöglichen. Dies würde nicht nur die Leistung der Klassifikation weiter verbessern, sondern auch die Anwendbarkeit des Modells in realen Szenarien erhöhen.

Darüber hinaus könnten zukünftige Arbeiten die Erforschung und Implementierung von Transfer Learning Techniken umfassen. Durch die Nutzung vortrainierter Modelle könnte die Trainingszeit erheblich reduziert und die Genauigkeit weiter gesteigert werden. Auch die Untersuchung verschiedener Regularisierungsmethoden könnte dazu beitragen, Überanpassungen zu vermeiden und die Generalisierungsfähigkeit des Modells zu erhöhen.

Ein weiterer interessanter Ansatz wäre die Implementierung von Ensemble-Methoden, bei denen mehrere Modelle kombiniert werden, um die Klassifikationsergebnisse zu verbessern. Dies könnte insbesondere bei komplexen Datensätzen von Vorteil sein, bei denen einzelne Modelle möglicherweise nicht die bestmögliche Leistung erzielen.

Zusammenfassend lässt sich sagen, dass die in dieser Arbeit entwickelten Methoden und Modelle eine solide Grundlage für die Bildklassifikation bieten. Die vorgeschlagenen Erweiterungen und zukünftigen Forschungsrichtungen haben das Potenzial, die Leistung und Anwendbarkeit der Modelle weiter zu steigern und neue Anwendungsbereiche zu erschließen.

Zukünftige Arbeiten sollten sich darauf konzentrieren, die Integration des Custom CNN in die Programmpipeline zu ermöglichen, um die Leistung der Klassifikation weiter zu verbessern.