
\chapter{Fazit und Ausblick} \label{chap:fazit_und_ausblick}

In dieser Arbeit wurde ein bestehendes Projekt zur Bildklassifikation von Leiterplattenfehlern weiterentwickelt und evaluiert. Der Schwerpunkt lag dabei auf der Implementierung einer Schnittstelle zur besseren Verwendbarkeit und Präsentierparkeit des bestehenden Projekts. Darüber hinaus wurden verschiedene Modelle und Methoden zur Bildklassifikation anhand eines neuen Datensatzes untersucht und miteinander verglichen.

Trotz der erzielten Fortschritte gibt es mehrere Bereiche, die für zukünftige Arbeiten von Interesse sein könnten. Eine mögliche Erweiterung besteht darin, die Integration des Custom CNN in die Programmpipeline zu ermöglichen. Darüber hinaus könnten weitere Modelle und Methoden zur Bildklassifikation untersucht und verglichen werden, um die bestmögliche Leistung zu erzielen.

Ergebnis ist eine Funktionierende Webanwendung, die es ermöglicht, neue Bilder des FESTO \ac{cp-lab} darzustellen, zu klassifizieren und die Ergebnisse auf einer Weboberfläche zu visualisieren. Die Anwendung koordiniert die Überwachung eines Ordners für neue Bilder und die Klassifizierung dieser Bilder in defekt oder nicht defekt. Die Implementierung erfolgte in Python unter Verwendung der Flask-Bibliothek für die Webanwendung und der TensorFlow-Bibliothek für die Bildklassifikation. Der Quellcode für dieses Projekt ist auf GitHub verfügbar und kann von Interessierten eingesehen und weiterentwickelt werden. Der Link zu diesem Repository und einer Anleitung ist im Anhang \autoref{subsec:anleitung_zur_verwendung} zu finden.

Eine weitere mögliche Fortsetzung dieser Studienarbeit wäre die Implementierung von Ensemble Methoden, bei denen mehrere Modelle kombiniert werden, um die Klassifikationsergebnisse zu verbessern. Dies könnte insbesondere bei komplexen Datensätzen von Vorteil sein, bei denen einzelne Modelle möglicherweise nicht die bestmögliche Leistung erzielen.

Zukünftige Arbeiten sollten sich darauf konzentrieren, die Integration des Custom CNN in die Programmpipeline zu ermöglichen, um die Leistung der Klassifikation weiter zu verbessern.