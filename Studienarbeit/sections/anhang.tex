\appendix
\renewcommand{\thesection}{\Alph{section}} % Nummerierung mit Buchstaben
\renewcommand{\thesubsection}{\thesection.\arabic{subsection}} % Zweite Ebene mit Zahlen
\renewcommand*{\sectionmark}[1]{\markright{Anhang \thesection: #1}}

\chapter{Anhang}
	
\subsection{Anleitung zur Verwendung} \label{subsec:anleitung_zur_verwendung}

Nach der korrekten Installation des Programmes, welche in Kapitel \ref{sec:installation} beschrieben ist, kann das Programm wie folgt verwendet werden:
Die Installation ist zum Zeitpunkt der Übergabe dieser Arbeit bereits auf dem FESTO-PC erfolgt. 


Zunächst sollte auf folgende Punkte geachtet werden:
\begin{itemize}
    \item Die Installation ist auf dem FESTO-PC erfolgt. Das Passwort ist bei frau \BetreuerDHBW \ zu erfragen.
    \item Die Verwendete Python Version ist 3.12.9.
    \item Alle verwendeten Bibliotheken sind in der requirements.txt Datei aufgelistet.
    \item Das Programmverzeichnis mit allen Daten liegt unter: 
    
    \texttt{D:/Repo/DHBWMA\_HHAT\_Image-Classification}
\end{itemize}

Um das Programm auszuführen, bitte der folgenden Anleitung folgen:

\begin{enumerate}
    \item Das PowerShell Skript auf dem Desktop mit dem namen \texttt{Image\_classification.ps1} ausführen
    \item Sobald in Powershell die URL \texttt{http://127.0.0.1:5000} angezeigt wird, kann die Webseite im Browser geöffnet werden.
    \item Wenn mit der VISOR Kamera der FESTO \ac{cp-lab} Kamera Fotos Archiviert werden, sollten diese auf der Website mit Klassifizierungsergebnis dargestellt werden. 
\end{enumerate}

Im Folgenden wird erklärt, wie Fotos mit der VISOR Kamera Archiviert werden können:

\begin{enumerate} 
    \item Die FESTO \ac{cp-lab} Anlage muss mit Strom und Druckluft versorgt sein.
    \item Auf dem FESTO-PC die Visor Software starten.
    \item Zunächst Config öffnen, das Passwort ist \texttt{Admin} für den Benutzer \texttt{admin}.
    \item Dieses Fenster kann direkt wieder geschlossen werden. (die Einstellungen sind bereits gespeichert, also \texttt{Yes} drücken)
    \item Im Visor Fenster auf \texttt{View} Klicken und unten links \texttt{Archiving} durch klicken aktivieren.
    \item Nun sollte die Kamera Fotos aufnehmen und diese in folgenden Ordner ablegen:
    
    \texttt{D:/Repo/DHBWMA\_HHAT\_Image-Classification/Bilder/new}

\end{enumerate}